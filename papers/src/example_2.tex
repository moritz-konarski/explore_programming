\documentclass[
    a4paper,
    12pt,
    man,
    donotrepeattitle
]{apa6}

\usepackage[american]{babel}
\usepackage[utf8]{inputenc}
\usepackage[
    threshold=40, 
    thresholdtype=words
]{csquotes}
\usepackage[hidelinks]{hyperref}
\usepackage[
    style=apa,
    sortcites=true,
    sorting=nyt,
    isbn=true,
    url=true,
    doi=true,
    hyperref=true,
    backref=false
]{biblatex}
\usepackage{mathptmx}
\usepackage[pages=some]{background}
\usepackage{wallpaper}

\renewcommand{\baselinestretch}{2.1}

\DeclareLanguageMapping{american}{american-apa}

\let \citeNP \cite
\let \citeA \textcite
\let \cite \parencite

\addbibresource{../bibliography.bib}

\newcommand{\mainheader}{Stories From The Homestead:\\\vspace{-12pt}An Environmental
                        Perspective}
\title{\textbf{\mainheader{}}}
\shorttitle{Stories From The Homestead}
\author{Montana B. Reid}
\affiliation{\vspace{-12pt}Anthropology Department,\\
    \vspace{-12pt}
    American University of Central Asia\\
    \vspace{-12pt}
    \today
}

\begin{document}

\ThisCenterWallPaper{0.81}{../title_background}

\backgroundsetup{
scale=1,
color=black,
angle=0,
opacity=1,
position={3.15in,-2.89in},
%placement=center,
contents={
    \setlength{\fboxsep}{-1pt}%
    \setlength{\fboxrule}{2pt}%
    \fbox{\includegraphics[width=3.3in,height=2.3in]{../white_background}}
  }%
}
\BgThispage

\maketitle

\tableofcontents
\vspace{12pt}
\newpage

\section*{\mainheader{}}\addcontentsline{toc}{subsection}{Introduction}

The history of my family and the land is a sordid collection of stories
stemming from forgotten homelands and fragmented remembrances.
I cannot possibly relate them
chronologically, or in any order other than the way I learned them. I learned
the history of my family, my history, in the late night talks of older
generations, in long drives through the mountains, and through faded
photographs. I cannot attest to how much is true in the fact-checkable sense,
only that it hardly matters as the stories, embellished or not, contribute to
how I understand myself, the past, and the land around me. As a young adult
reasonably separated from my family it would be great to write an essay
explaining how far I have come from the ideas endemic in the environment I grew
up in, but that is not possible. I am a product of those who came before and
I see the land the way they did. To put my family into context, none of them 
were
ever very rich, none ever lived in cities larger than 100,000 people and all of
them consider nature to be part of who they are.

\subsection{The Story of a Logging Camp}

Oregon is the home of my family since 1865. I know this because, disbelieving,
I went to the Oregon historical society to check, and there, in the dusty card
catalogue, I found the name "Phoebe Applebee Taylor". The Taylors as it turns
out are a founding family of the state and if I was so inclined I could drive
three hours to their cemetery; the place where the Taylors forever joined the
state of Oregon. Logging is and was one of the largest industries in the state.
There used to be vast forests of virgin timber, now it is a rather large-scale
tree farm. My great grandmother was born in a logging camp in 1925. She lived
in a house with a beaten dirt floor, a wood burning stove and an outhouse. Her
father, uncles, and brother were loggers. Her father lived and died in the
logging industry. One day at lunch he removed his hard hat to have his
sandwich and was hit with what is called a "widow maker" (a branch from
a three that was felled nearby), he died almost instantly.

My great grandmother is a die-hard democrat. When I asked her why she said
because she lived through The Depression, she knew what it was to suffer, what
it was to receive no help from the government. She used to tell stories of her
and her brother hunting as children in the logging camp. To catch a rabbit or
squirrel meant they would get to eat some meat. Despite her democratic views,
she is not what I would consider an environmentalist. She is pro-logging. While
lack of worker safety and rights is something she rants on constantly, she also
remembers a time before regulation. The fact that \blockquote{transnationals head this
drive for no national or international regulation, responsibility or
accountability} \cite[p. 72]{s3}, is something that would be abhorrent to her, and yet
she is the sort of person who would agree with \blockquote{wise use} \cite[p. 88]{s3} policies. If
she turned on the news though as she does every night and saw that
a \blockquote{US timber
firm, Louisiana Pacific moved part of its operations to Mexico to exploit cheap
labour, it told its workers that environmentalists were to blame for the job
losses} \cite[p. 94]{s3}, she would believe it. She would talk about how
"environmentalists" had hurt poor workers. She would identify with the poor
workers working a dangerous job and not really think about the company that was
making the profits.

In her lifetime she saw the last of the virgin timber cut, the industry slowly
fade, but the stories that stay with her are not the lack of trees, it is
things like the controversy of the spotted owl. The spotted owl is a rather
scruffy-looking little bird that dealt a blow to the logging industry in the
time of my parents. It is not that this tiny bird itself represented much of
a threat but it's near annihilation caused much a stir. The owl's habitat
was being destroyed by the logging industry despite their tepid attempts to
regrow the forest and mitigate habitat loss by replanting areas they chopped 
down. The spotted owl
nested in large trees, old growth, something there was almost nothing left of
by that time. Environmentalists at the time campaigned very hard to save the
owls and eventually were able to have restrictions put in place to stop logging
in many areas. This ultimately slowed down the industry. But how does this 
seem to
a woman who was a child in the Great Depression? To her logging equaled food 
and shelter. Why should she care about an owl or environmentalists? They do not
understand the land, the trees or what logging represents.

\subsection{The Toughest Man in the Rouge Valley}

Jack Newton was the toughest man in the Rouge Valley and my great grandfather.
He was a man born of the Old West and lived his life as a hard drinking and
gambling poacher. I know his stories through my father who revered him; to my
disadvantage he was an old sick man for most of my life. I remember him mostly
sitting in his recliner, ignoring my great grandmother. The first story about
him is how he became the toughest man in the Rouge Valley. According to my
grandmother (his daughter), one night he went out drinking in a local bar (as
he often did) and got in a fight with a man. He beat the man and went home, the
man as it turned out was the well known and widely accepted "toughest man in
the valley". The next day the man showed up at their house. He demanded that
Jack fight him again, the night before he said he was drunk, so it was not
a fair fight. My great grandfather went out onto the front lawn and proceeded
to fistfight the man. He beat him again. My grandmother told me she watched
from the window, just a girl at the time. The man went home, head hung low and
Jack Newton became known as the toughest guy in the valley. I never quite
believed it, and was always too shy to ask, but according to my father Grandpa
Jack had his nose broken so many times that they took the cartilage out and
replaced it with rubber. This meant that he could bend his nose flat to his
face.

In the words of \textcite[p. 12]{s10}, \blockquote{any wildlife management ran 
counter 
to American ideology.} My Grandpa Jack was a lot of things, 
one of them was
a poacher. He did not believe the government had the right to tell him how to
use the forest. The idea of restriction was simply un-American. He poached his
whole life, through The Depression, upon his return from WWII, and into his
later years when he took my father hunting. He poached to feed his family, his
drinking, and his ego. He poached not just for those close to him though, he
also sold meat. He sold venison to a local restaurant for many years. The
restaurant called Omar's still stands to this day and we often take my great
grandmother (his wife) there on special occasions. Just as those who shot for
mere sport looked down on market hunters, market hunters looked down upon them
\cite[p. 26]{s10}. The American western philosophy of doing things by the sweat
of your
brow pervaded through my family thanks to Grandpa Jack (whose real name is
actually Gerald, a fact I learned after his death).

Despite Grandpa Jack's disregard for any government policy, he loved and
worshiped nature. This may be hard to believe of a man who regularly poached
deer and sold Christmas trees he had stolen. The old school method for getting
a good Christmas tree was to find a big beautiful fir tree, lock it in your
sights and blast the top off with a shotgun. The top of the tree would fall to
the ground and be much fuller and nicer looking than any young tree you could
harvest. This of course was more than frowned upon, it was highly illegal,
mostly because it tends to kill what is left of the tree. This fact never
deterred Grandpa Jack from taking the trees down to Ashland to sell in time for
Christmas. To me it seems this constant usage and interaction with wilderness
is what created his love and respect for it. To my Grandpa Jack \blockquote{God
was on
the mountaintop, in the chasm, in the waterfall, in the thundercloud, in the
rainbow, in the sunset} \cite[p. 73]{s8}. This idea was inherited by my father
and he has
spent much of his life trying to "get back to nature", he feels comfortable
there. My father has lived his whole life in medium sized towns. He enjoys his
convenient grocery store and withers without Netflix. If one where to ask him
about his best memories though they all are somehow tied to the woods and my
Grandpa Jack. My father saw my grandpa as a dying breed and the woods as the
manifestation of \blockquote{rugged individualism} \cite[p. 77]{s8}. My great 
grandfather did as he
pleased and used the woods and by wider extension life how he wished. He took
little heed of law or changing times. His wildest stories revolve around the
brothels of Pendleton (a frontier, gold rush sort of town), where he acted in
the role of bodyguard for a prominent Native American healer. My father goes to
the woods to reconnect with this lost time and ideas. He thinks of himself as
living this sort of wild roguish mountain man life. The reality is closer to
the words of \textcite{s8}: \blockquote{To the extent that we live in an
urban-industrial civilization but at the same time pretend to ourselves that
our real home is in the wilderness to just that extent we give ourselves
permissions to evade responsibility for the lives we actually lead. (p. 81)} 

\subsection{The Dangers of Newspaper Editing}

\blockquote{Epidemiologists must be 95 percent sure before they will conclude 
that a correlation exists between exposure and disease} \cite[p. 77]{s1}. This 
leaves a lot of
room for people to fall through the cracks. This is the account of my paternal
grandfather, John Nobel Reid the second (and yes I do thank God everyday that
I was born a girl and am not named "the fourth"). My grandfather was diagnosed
with Parkinson's disease when he was only 35, the disease would ultimately take
his mind, his movement and when I was 16, his life. He was an intellectual man
to the last, his father was a prominent doctor, his mother was old money (with
a claim to being part Rockefeller, what part it was exactly she never said),
his brother is a lawyer and he was editor of the newspaper. He met his wife in
college, she was from the wrong side of town and was unfortunately a "Newton".
Despite the trepidation of his parents they married and had three children. He
started as a journalist and worked his way up to editor. There was on thing
however that would put a crimp in his plans to continue to rise and that was
the diagnosis. It would be easy to say from a modern and external perspective
that he had bad luck, or that he was genetically predisposed or the heavy
indoor smoking prevalent in that industry caused his disease, but that would
not be the whole story. For many years his office was this room, not a room
originally dedicated to the purpose but rather converted, it had been a room
for developing photographs. I am not familiar with all the chemicals used in
a dark room, but I know that they are harmful and noxious. It seems a wild
claim right? Where is the 95\% assurance that the two are linked? Where can one
prove that the newspaper is responsible for his chemical exposure and loss of
life? There is no study conducted, only anecdotal evidence; he was young and
healthy, no one else in the family history ever had Parkinson's, the man who
had the office before him also developed it, and they no longer use the room as
an office. Shouldn't this count for something? Shouldn't there be
responsibility? I am not claiming they did not treat him well, long after he
was able to work, they had him write editorials from home. I remember him
typing them on my grandmother's big, boxy, plastic-coated desktop, it was
bright blue and it always sat on a wooden desk full of little drawers. I also
remember her having to help him, or even write part of the editorials herself.
When I was quite small my grandfather had a brain surgery, they implanted an
electrical device in his brain to help with his tremors and movement. His father
the doctor was against it, it was new and controversial, his parents never
forgave him or my grandmother for going through with the surgery. It changed
him irreparably, his mind would never be what it once was. I never knew my
grandfather, though he died when I was a teenager. His mind was gone, I knew
him as the man who ate Almond Roca and accidentally stepped on the dogs. I was
shocked to learn he was a published poet, I never knew that man. A high burden
of proof may be needed for companies to believe they are hurting people, but
for my family, for me, the biggest proof is what we all watched happened the
slow, painful, decline of a promising man.

\subsection{Sportsman's Warehouse and the Dump}

This is a story about waste and the commercialization of nature. Both my father
and stepfather consider trips to the sporting goods store to be a valid and
exciting way to spend the afternoon. They seldom bought, mostly just sighed
over new "toys" they couldn't quite buy. In the same way that author Jennifer
Price saw the Nature Store as a way to sell the experience and culture of
nature, without nature being involved \cite[pp. 186--196]{s9}, my father and 
stepfather actively visited the sporting goods store. My step father lived 
in the
countryside, my father hunted and fished regularly, both were often in contact
with real nature, yet it is to the sporting goods store they went.

My father's favorite store is Sportsman's Warehouse. He also has a soft sport
for a local store called Black Bird, I think mostly because for him it is
a connection to his childhood and the 50 foot tall bird statue out front makes
him smile. The corporate chain Sportsman's Warehouse however had the new
products, the selection, and the things he never knew he needed. It is an
experiential place: there are boats on the walls, a wide array of taxidermied
animals from around the country, and there is a section where they have set up
tents and you can go inside and look around (my favorite part of the trip as
a child). There is a sense of community, with bulletin boards displaying
pictures of local hunters and what they "bagged". There is even a counter where
you can buy game tags and hunting/fishing licenses. It sells the complete
outdoorsman's package: clothes, equipment, guns, knives, safes, shoes, food,
and most importantly an idea. They sell the idea of independence, of comradery,
and that the man who has the most gear is the best. This idea is problematic
because it connects nature with having stuff, and stuff leads to waste.
In the words of \textcite[p. 149]{s7}, \blockquote{raising prosperity affects 
the environment. The more objects people buy and discard, the more waste is 
created}. The trends in outdoor equipment
are no less changeable than those in the fashion industry. Those who think
hunters and fishermen are simple country folk happy with a two-dollar garage
sale rod and a flannel are mistaken. They want the best, the newest, the thing
they can show their buddies. My father has a friend who got a very nice new
pickup truck, he bought a gun to match, it is complementary in color and
specially shortened to fit on the back of his seat. He does not need this to
ensure he will shoot a deer, it is an accessory, a fashion statement.

This culture creates an unbelievable amount of waste. The trip to the dump is
also ritualized in no less of a way than the trip to Sportsman's Warehouse. One
might need a trailer or to borrow some straps, maybe your friend goes with you
to help. There is bonding and respect of the amount you threw away. I went to
the dump several times as a child, mostly to drop off yard debris. More often
than not I was just required to help collect the debris and trash and not
actually taken to the dump. I was always excited about going, one because it
sounded like a crazy place and two because going out into the city always
implied the possibility of stopping somewhere for a treat. As a child much of
my energy was expended in the quest for ice cream, despite the fact I am
semi-lactose intolerant and it always made my stomach hurt. In my family the
connection between nature, the store, and the dump is firmly fixed. It moves
cyclically through the lives of the men and shapes who they are.

\subsection{Have a Fish Ladder}

\blockquote{An ethics that considered only effects on nature and ignored humans 
would be
irrelevant to the practical politics of environmental activism and would cut
itself off from real policy debates} \cite[p. 710]{s2}. This is key to the 
disagreement
between the environmentalists and the rest of the state. The issue in question:
fish. I caught my first fish when I was about seven, it is a very clear memory,
partly because it was retold to me so many times. It was during a visit to my
mother's only sister. My aunt lives in a secluded part of Northeastern Oregon,
high in the mountains in the town of Joseph. Joseph is a tourist town, mostly
for its beautiful lake and yearly celebrations commemorating Chief Joseph.
Before going to the lake, I informed all the adults that I would catch a fish
"this big" holding my little arms a good 20 inches apart. They laughed and said
"of course you will sweetie." We went to the rocky lake shore, I insisted
a nightcrawler and a green mini marshmallow be attached to my hook. Sure enough
not long later I caught a huge rainbow trout, bigger than my earlier boast. At
the time I had no conception that the lake had been stocked with trout or that
the management of fish in my state was hotly contested.

Similar to the \blockquote{various governmental legislation and regulations 
also impact
the business activities of the tuna TNCs} \cite[p. 161]{s5} the salmon industry
and the
hydroelectric industry faced problems. Salmon travel inland to small streams
across Oregon to spawn and die. They live most of their adult lives off the
coast of the Pacific Northwest. Salmon fishing is popular in my family. My
step-grandfather owns a sea-worthy boat and often goes to the ocean or river to
catch salmon. The law is you get to keep one, and not a native one. Most salmon
are from fish farms, and then released when they are juveniles. My step
grandfather would take a salmon and slow cook it for hours on his Traeger
barbecue. On the way to school every day could see if the salmon where running
in the rivers or not. The freeway goes over the Columbia River and there you
see hundreds of boats jammed together trying to catch one. There are a few
reasons why salmon (a well as other fish, like trout) are grown in fish
hatcheries. One is that overfishing lead to a decline in their numbers. Another
reason is the hydroelectric dams that sit on the main rivers. The fish can't go
through them really, salmon swim against the current to spawn and the dams have
giant blender-like turbines. The fish hatchery is a magical place, we went
there for a school trip in elementary school. There are big fish ponds full of
baby fish and you can feed them. One of the fins of each little fish is snipped
to ensure later in its life a fisherman will not mistake it for a wild fish.
The fish hatchery even has its own mascot. Herman the Sturgeon is a giant fish
maybe 15 foot long, with a stoic expression befitting a species as old as the
dinosaurs. There was a scandal with poor Herman, teenage hooligans broke into
the hatchery at night and savagely murdered him in his tank. He had been
stabbed several times. There was an outpouring of public mourning for the poor
fish.

The environmentally conscious city of Portland clashed with the rural areas of
the state. They wanted to end fishing in general, to allow the populations to
regrow. There was a problem still though, the hydroelectric dams. The state
needed the dams for cleaner energy and the rural areas needed the revenue
fishing tourism provided; thus, a semi-effective solution was found: a fish
ladder. It sound ridiculous, how could a fish use a ladder? Well, they are more
like steps with water rushing down them. They effectively allow salmon to
bypass dams and reach the place of their birth. By and large this works and the
number of salmon has risen dramatically in recent years. There is a slight
problem with the ladder though, but it is mostly funny. In the large rivers
only a few hours from the ocean there is a fairly healthy population of seals.
Now seals are not the most athletic or smart creatures but nonetheless they
figured out the fish ladders. They are too big to go up the ladders but they
wait around the bottom of the ladder and eat all the fish. I have heard a rumor
that there are now government workers whose job it is to scare away the
gluttonous seals, but I have never been able to confirm it.

\subsection{AM Radio Told Me So}

AM radio is the music of my childhood. I have never seen my father turn it back
to FM, in fact when I started driving and we would share a car, he would
angrily switch the radio over to AM as soon as he turned the key. I'm not sure
if this stems from a particular taste in music or a real love of talk shows;
regardless it was always on. Now I simply tune out the shows or talk over them.
They are by and large conservative talk shows, peppered with a gardening advice
shows and "Coast to Coast". When I was younger I listened and tried to
understand, this is how I know Rush Limbaugh. His controversial and
conspiratorial rhetoric is too much to express in one story, but as far it
relates to global warming and pollution it is possible to summarize. In the
words of \textcite[p. 145]{s4}, \blockquote{environmentalism and 
environmental science are two of Limbaugh's favorite topics for dour 
denunciation}. Limbaugh thinks global warming is a hoax, stating: 
\blockquote{the fact is we couldn't destroy the earth if we wanted to} 
\cite[p. 145]{s4}.

These ideas and conspiracy theories in general are widely accepted in my
family. As a group they believe the government should not be trusted. My father
and his sisters are all fairly sure Bush created 9/11. It would be wrong to
suppose that AM radio makes them conservative, they are not conservatives. The
truth is they are anti-large group and eagerly await the day society collapses
and it becomes open season on the politicians, Republican and Democrat alike.
There is of course a limit, no one wears tinfoil hats or excludes themselves
from society at large. My aunt Erin does not believe the moon landing really
happened and she is mercilessly teased by her siblings for it. The general
skepticism in anything they are told however becomes apparent in the
relationship between global warming and pollution. This is exemplified by this
excerpt from \textcite{s6}:
\blockquote{Paul Slovic, an American professor of psychology once asked for 
groups of
people to rank various risks. A group of experts put nuclear power twentieth in
order of risk, well below surgery, x-rays and private aviation. A group of
business men executives ranked it eight. Two groups, members of the League of
Woman Voters and college students, rated nuclear power worst of all. (p.
139)}
The quote demonstrates the power of perception upon the public. In my family
the power of perception is no different. An easy example of this can be found
in global warming vs. nuclear pollution. My family in general doesn't believe
very much in global warming, maybe from the talk shows but more likely because
it is the government and scientists who tell them to be concerned. They are
people seen as not to be trusted and having a possibly nefarious agenda.
Nuclear pollution by contrast is accepted a serious problem. After the
earthquake in Japan lots of debris, some of it from the nuclear plant, ended up
dumped into the Pacific Ocean. For years after the earthquake items from Japan
washed up on Oregon beaches. There used to be a tradition of looking for
glass-floats from Japan and China when my grandmother was a girl. In my own
childhood we searched for smoothed pieces of beach glass and imagined exotic,
far off origins. This however was not the sense of nostalgia that accompanied
the barrels of toxic waste that washed ashore. Maybe it was nostalgic for those
from the Cold War generation, or those who lived on the coast during WWII when
Imperial Japan actively tried to burn down the forests. There was no trouble
believing this posed a danger, there was no complaint about respecting closed
beaches. This is something that was witnessed, and a danger that is held in the
collective cultural consciousness. Global warming has none of that, they say
the summer is cooler than it was last year. How could it be warming? AM radio
and nut jobs like Rush Limbaugh contribute to a general mistrust in my family.
They want to see things, know things for themselves, regardless of what the
radio or anyone else might say.

\subsection{Cowboys and Indians}

It has come full circle in the story of my family. At the beginning once again,
it is the time of ethnic uncertainties and frontiers. What I know of the racial
background of my family is confusing. Two years ago my aunt took
an ancestry.com DNA test that left us with more questions than answers, so
I will stick with the party line, the story I was told. I am the product of
every type of European, my last name is Scotch-Irish. It is assumed we came
over during the Irish Potato Famine, but it is hard to prove, the name John
Reid is almost as common as John Smith in the records. On my mother's side
things go back much further, to before the country declared independence. In
her family there are many famous names thrown around. We are related to Thomas
Jefferson (isn't everyone?) and more plausibly related to Alexander Graham Bell
the inventor of the telephone. This claim is believable simply because we are
also Bells and the name was given to me in the form of a middle name.

What is missing from these last names and vague percentages is the Native
American aspect. On my mother's side it is fairly verified, a picture of
a distant grandmother (great great or great, great, great?) in the traditional
clothes of her people on her wedding day. On my fathers side it was more
hidden. Not hidden in a clear or even original way, but in the way of so many
families in this region. My paternal great grandmothers said three different
things. My grandma Anne said she was "Dark Swiss", or sometimes "Black Dutch",
these names are silly and obviously not real groups of people. Then there is my
great grandma Phyllis, who claims her father was part Spanish. This is
a possibility but not a probability. He was Mexican or Native American there
were no Spaniards running around Oregon at that time. There is the also
obvious evidence, she looks like an old Native American woman, and tans very
well despite the rest of us being white as moonlight.

I remember the first time I heard I was part Native American. I was quite
young, maybe six and I was snuggled in my grandmothers big bed. I was watching
cartoons and it was a warm summer morning, I remember all the tiny perfume
bottles she always kept on her dresser. My grandma Karen had been playing with
my hair, trying to smooth it down, then she stopped and touched my cheekbones.
She said "see Tana, you have high, beautiful cheekbones like your grandma, it
is because you are part Indian." I remember looking at her cheekbones, a bit
hidden in her soft wrinkly face and wondering how a cheekbone could be high?
How high was too high? And could I move them lower if I tried hard enough?

The Native American connection in my family most likely happened when they
moved to Oregon. My fathers family crossed the Oregon trail in the 1860s and we
still have many artifacts brought along the trail. The only thing I personally
have is a tiny black wood box with a golden latch, inside is a tiny set of
weights and a scale. I am told it was used by my ancestor to weigh gold when he
came across the trail, and that he was a conductor of wagon trains. My aunt
owns a dress beautifully hand-sewn and brown with age that was made by Phoebe
Applebee Taylor on her way to Oregon to be used as her wedding dress. We have
a lot of Native American artifacts also but they are not from family members.
The woven baskets and stone mortar and pestle were given to my great great
uncle Lynn, who died when my father was a young boy. Uncle Lynn was a doctor
and he would often go to the reservation to treat people. The people living on
the reservation were very poor and often asked to pay with goods, he always
agreed.

The life of Native Americans from the westward expansion to the present has
been rife with sorrow. Now on many reservations the biggest killer is alcohol
and drug abuse. Like any economically depressed minority in the United States
they face little opportunity for expansion or better lives. Having a degree of
self governance and enactment of separate law systems is a blessing, however
there are un-seen drawbacks. In Pokey's Paradox it states that 
\blockquote{tourism is a devils
bargain, a choice between change and the remaking of sociocultural lines or
stasis and the ongoing poverty and marginality that accompany the lack of
strategies for change} \cite[p. 93]{s11}. In Oregon the barbed gift of tourism 
arrives in
the form of casinos. They are giant multimillion dollar places. They are
successful because gambling is illegal everywhere but on the reservation or
other tribe owned lands. They sell an image of their culture chopped up and
commercialized. My great grandma Phyllis loves the casino, she is always asking
if we will drive her the two and a half hours to "The Feathers". Seven Feathers
is a casino, hotel, and event center not too far from the city of Roseburg. It
sits on tribal lands and while the casino is state of the art the tiny town
attached to it is economically depressed and lost in time. I can't help but
wonder how my great grandmother would feel if she were willing to acknowledge
her Native American heritage. Would she enjoy the "slots" quite so much? Maybe,
but I seriously hope not.

\subsection{My Understanding}

The history of my family and the land is a big part of my story, but not all of
it. I was born in 1997 to parents who lived on rural property. From the start
I was unbelievably stubborn and choose to make my own way in life. I grew up in
the trees, making mud pies, and sword fighting with ferns. I respect the
traditions of my family, but I have chosen not to be so tied to the land. I am
a 6th generation Oregonian and I will not have my children in the state. I want
to connect to nature and culture on a wider scale. My final opinions about
environmentalism are not yet set. I think the reason can be found in the word
itself, the word environment. It is the place where you live, you grow, you
exist; how can I understand environmentalism when I don't understand my current
environment? This collection of stories, of memories, has explained more to me
about my own understanding than that of my family. They may disagree with that
I have written, but regardless it is my understating of the environment, it is
perspective that shapes us. I hope it is through an ever evolving perspective
that I can view the world, nature, environmentalism, and my family.

\printbibliography\addcontentsline{toc}{subsection}{References}

\end{document}
